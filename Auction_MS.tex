\documentclass [man]{apa6}
\usepackage{graphicx}
\usepackage{multicol} 
\usepackage{fullpage}
\usepackage{amssymb,amsmath}
\usepackage{fullpage}
\usepackage{float}
\usepackage[lofdepth,lotdepth]{subfig}
\title{Social influnces and their interaction with private values in first bid all pay auctions  }
\date{}
\author{Toelch, Jubera-Garcia \& Dolan}
\shorttitle{Social influencesin auctions.}

\begin{document}
\maketitle

\section{Introduction}
To make informed decisions individuals need to sample their environment to update prior knwoledge to adapt their actions to prevailing circumstances (Dall...). It is thus not surprising that individuals integrate over personal and social information sources to reduce posterior uncertainty (Rendell TICS; Toelch). Even though the use of inadvertent social information is adaptive under a wide range of conditions (Rendell Science paper, McNamara 2009) the integration process can also lead to suboptimal decision making as seen in infromational cascades (bikchandri).  This translates further to competitive social contexts, e.g. sealed bid first price auctions, where competitors often overbid the actual value of a resource resulting in the so called winner's curse (Thaler 1988). The classic example comes from the sale of oil drilling leases in the gulf of Mexcio where bidders in many cases paid more for a lease than the discovered oil was worth (Thaler 1988). That is, even though companies won the drilling lease they would generate negative revenue.\\
These effects were also discovered in laboratory experiments (Bazerman Samuelson 1984) (Cooper/Fang; Charnes/Levin).  



\section {Methods}
\subsection{Participants}
Participants were recruited from the local participant pool via email invitation. In total 42 (17 male) participants played the game in pairs of two with a maximum of four players per session. Players could freely choose their computer at the beginning of the session which resulted in 10 same gender pairs and 11 mixed gender pairs. All procedures comply with APA guidlines and the declaration of Helsinki. 
\subsection{Auction Game}
Players played a first-bid all pay auction game for five different real items in pairs. Prior to playing the actual game participants received a training of 20 rounds to familiarise them with the controls and the mechanics of the game. During this training the five auction items were replaced by abstract figures. After training players were given the opportunity to inspect the available auction items. All items (candle, pens, box of chocolate, one way camera, herbal tea) were purchased at approximately the same price (4.5 to 5.5 Euro). The price of the items was not revealed to the participants. After inspection players ranked the items according to their preference. Participants played 200 auctions (40 for each item) randomly interspersed. In each round players were allotted 100 points. These points had to be distributed to either the auction item or to a monetary lottery with a price of 7 Euro, which was higher than the actual cost of each item. The player with the highest amount of points would win the auction in this round. The points allocated to the lottery (divided by 100) represented the chance to win 7 Euro in this round. An example; Two players bid for an item. Player 1 bids 25 points and player 2 bids 40 points. In this round player 2 wins the item and has an additional chance of 60\% to win 7 Euro. Player 1 does not win the auction but has a 75\% chance to win the lottery. At the end of the game participants had to rank the items again for their preference. One round was randomly selected for each player and the actual outcome was paid to each participant, that is, participants could actually win one of the items and additionally 7 Euro. Participants who did not win either received 3 Euro. All participants received a show-up fee of 5 Euro. To assess the private value of each item of each participant feedback was not provided in the first five rounds of the experiment where all five items were presented. In all other rounds participants received feedback on whether they won the auction but not the lottery and how much the other player bid for the item. 
\subsubsection{Manipulation of preferences} Since we were interested to explore the interaction between private value and social influences we performed a manipulation on the items players saw in each round by matching preferences of players in the auction. We ordered items via the preferences participants gave prior to the auctions. A pair of players would bid on the item with the same preference, which was not necessarily the same item. For example, player 1 chose the candle as third preference and player 2 chose the pens as third preference so in one round player 1 would bid for the candle while player two would bid for the pens. To also create conditions with high differences between the two initial bids we also switched items of preference 2 and 4 for one of the two players. This resulted in player 1 seeing the item with the second preference and player 2 seeing the item with the fourth preference. This effectively created a high and a low competition and three equal competition conditions. Players were not informed about this manipulation.
 

\section{Results}



\section{Discussion}



\bibliography{mybib}{}
\bibliographystyle{apa6}

\end{document}